%%%%%%%%%%%%%%%%%%%%%%%%%%%%%%%%%%%%%%%%%%%%%%%%%%%%%%%%%%%%%%%%%%%%%%%%%%%%%%
%%
%% Author: zeta709 (zeta709@gmail.com)
%% Releasedate: 2017/07/20
%%
%% 목차 양식을 변경하는 코드
%% * subfigure (subfig) package 사용 여부에 따라
%%   tocloft의 옵션을 다르게 지정해야 한다.
%% * Chapter 번호가 두 자리 수를 넘어가는 경우 다음과 같이
%%   필요한 만큼 "9"를 추가하면 된다.
%%   \settowidth{\mytmplen}{\bfseries\cftchappresnum9\cftchapaftersnum}
%%   아니면 \cftchapnumwidth를 직접 적당히 고치면 된다.
%%%%%%%%%%%%%%%%%%%%%%%%%%%%%%%%%%%%%%%%%%%%%%%%%%%%%%%%%%%%%%%%%%%%%%%%%%%%%%
%\usepackage[titles,subfigure]{tocloft} % when you use subfigure package
\usepackage[titles]{tocloft} % when you don't use subfigure package
\makeatletter
\if@snu@ko
	\renewcommand\cftchappresnum{제~}
	\renewcommand\cftchapaftersnum{~장}
	\renewcommand\cftfigpresnum{그림~}
	\renewcommand\cfttabpresnum{표~}
\else
	\renewcommand\cftchappresnum{Chapter~}
	\renewcommand\cftfigpresnum{Figure~}
	\renewcommand\cfttabpresnum{Table~}
\fi
\makeatother
\newlength{\mytmplen}
\settowidth{\mytmplen}{\bfseries\cftchappresnum\cftchapaftersnum}
\addtolength{\cftchapnumwidth}{\mytmplen}
\settowidth{\mytmplen}{\bfseries\cftfigpresnum\cftfigaftersnum}
\addtolength{\cftfignumwidth}{\mytmplen}
\settowidth{\mytmplen}{\bfseries\cfttabpresnum\cfttabaftersnum}
\addtolength{\cfttabnumwidth}{\mytmplen}
\makeatletter
\g@addto@macro\appendix{%
	\addtocontents{toc}{%
		\settowidth{\mytmplen}{\bfseries\protect\cftchappresnum\protect\cftchapaftersnum}%
		\addtolength{\cftchapnumwidth}{-\mytmplen}%
		\protect\renewcommand{\protect\cftchappresnum}{\appendixname~}%
		\protect\renewcommand{\protect\cftchapaftersnum}{}%
		\settowidth{\mytmplen}{\bfseries\protect\cftchappresnum\protect\cftchapaftersnum}%
		\addtolength{\cftchapnumwidth}{\mytmplen}%
	}%
}
\makeatother
